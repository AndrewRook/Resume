% LaTeX file for resume 
% This file uses the resume document class (res.cls)
\documentclass[11pt]{res}
\usepackage{hyperref}

% \topmargin=-0.380in  % start text higher on the page
% \setlength{\textheight}{9.75in} % increase text height to fit resume on 1 page
\newsectionwidth{0pt}  % So the text is not indented under section headings
\begin{document}  
\name{Andrew Schechtman-Rook}

\address{rook166@gmail.com\\(917)-836-4267}
\address{https://github.com/AndrewRook}
                        
\begin{resume} 
\section{\centerline{\large TECHNICAL SKILLS}}
                    {\bf Programming:} Python (numpy, scipy, pandas,
                    sklearn, xgboost, matplotlib), shell \nobreak{scripting}\\
                    {\bf Model Development \& Deployment:} GBMs, random forests, linear \& logistic regression, \nobreak{nonlinear} optimization, parallel \& distributed computing, containerization, APIs\\
                      {\bf Databases, Orchestration, Web Design:}
                      Flask, MySQL/PostgreSQL, Airflow, Prefect, Snowflake\\
                      {\bf Cloud Computing/Devops:} AWS, CircleCI, Jenkins, GitHub Actions
\section{\centerline{\large RELEVANT PROJECTS}}
\vspace{0.03in}
{\bf \href{https://github.com/AndrewRook/ptplot}{Ptplot: https://github.com/AndrewRook/ptplot}}
\vspace*{0.02 in}\begin{itemize} \itemsep -2pt
  \item Python package for quickly creating interactive visualizations of player tracking data.
  \item Visualizations can be viewed in notebooks or embedded in websites.
\end{itemize}
\vspace{-0.15in}
{\bf \href{https://andrewrook.github.io/NFLDash}{NFLDash: https://andrewrook.github.io/NFLDash}}
\vspace*{0.02 in}\begin{itemize} \itemsep -2pt
  \item Web dashboard allowing users to interactively filter events on 15+ dimensions, then export results to disk.
  \item Built in HTML and Javascript with d3.js and crossfilter.js. 
\end{itemize}
\vspace{-0.15in}
{\bf \href{https://github.com/AndrewRook/NFLWin}{NFLWin: https://github.com/AndrewRook/NFLWin}}
\vspace*{0.02 in}\begin{itemize} \itemsep -2pt
  \item Python package for calculating Win Probability of NFL play-by-play data.
  \item One of the first fully open (code \& algorithm) published Win Probability models.
\end{itemize}

\section{\centerline{\large WORK EXPERIENCE}}
{\bf Director, Data Science:} Capital One\hfill\mbox{2023-Present}\\
{\it Technical lead for Card Credit Innovation, building and maintaining core credit infrastructure}
\vspace*{0.01 in}\begin{itemize} \itemsep -2pt 
  \item Designed and built a Python package to streamline core business
    metrics SQL calculations, used by over 50 analysts for critical
    reporting needs. 
  \item Prototyped an LLM-based approach to automating SQL query generation using RAG, with accuracies up to 80\% on test datasets.
  \item Mentored multiple data scientists, including starting an individual-contributor focused talk series.
\end{itemize}

{\bf Senior Manager, Data Science:} Capital One\hfill\mbox{2018-2023}\\
{\it Model developer and technical lead for Upmarket Card Data Science and core developer on valuation model infrastructure team}
\vspace*{0.01 in}\begin{itemize} \itemsep -2pt
  \item Guided ongoing development of the core credit card valuations model scoring platform, delivering regular releases of new and updated models while improving the robustness and maintainability of platform infrastructure. 
  \item Led technical development of model monitoring tools, mentoring three junior data scientists to deliver a maintainable package on time and to spec.
  \item Deployed the first cloud-based credit card underwriting model
    in the company via a dockerized Python API, with an estimated
    incremental value of 35 million dollars per year.
\end{itemize}

\pagebreak
{\bf Manager, Data Science:} Capital One\hfill\mbox{2016-2018}\\
{\it Underwriting model deployment and tooling subject matter expert}
\vspace*{0.01 in}\begin{itemize} \itemsep -2pt
  \item Led development of a prototype language-agnostic automated machine learning model
    deployment framework for cloud-based applications, influencing the
    development direction for the company-wide credit card application
    processing platform.
  \item Created the longest-lived, most successful
    internal data science tool in the company, used in production models by dozens of
    data scientists across multiple lines of business.
\end{itemize}

{\bf Principal Data Scientist:} Capital One Labs\hfill\mbox{2014-2016}\\
{\it Core performer and project manager for an internal data science training program}
\vspace*{0.01 in}\begin{itemize} \itemsep -2pt 
  \item Implemented a novel approach to deliver internal technical
    trainings, providing over 5000 hours of classes with no instructors.
  \item Programmed and deployed an interactive course completion dashboard using Flask
    and dc.js to provide progress reports to individual students as
    well as company leadership.
  \end{itemize}

{\bf Research Associate:} University of Wisconsin-Madison\hfill\mbox{2014}\\
{\it Postdoctoral researcher}
                  \vspace* {0.01 in}\begin{itemize} \itemsep -2pt
                    \item Devised metrics to improve correspondence between numerical models and
                      astronomical data. Implemented in highly
                      optimized Python, was able to refine agreement by
                      up to 20\% with minimal increase in computation time.
                    \item Built a fast Voronoi Tessellation algorithm
                      to adaptively bin images, preserving spatial
                      resolution while maximizing signal in images
                      with over one million pixels.
                    \item Trained and mentored undergraduate and
                      graduate students in programming, data analysis and statistical methods.
                    \end{itemize}

{\bf Research Assistant:} University of Wisconsin-Madison\hfill\mbox{2007-2013}\\
{\it Graduate student}
                  \vspace* {0.01 in}\begin{itemize} \itemsep -2pt
                    \item Developed non-linear Levenberg-Marquardt
                      $\chi^{2}$ fitting algorithms using a
                      combination of Python and C++ to constrain
                      models of spiral galaxies to data. 
                   \item Employed on-campus distributed computing
                     resources to perform large-scale modeling in parallel, using over 20 years of 
                     computer timein 1 month.
                   \item Assembled a hybrid C++/Python processing
                     pipeline to clean, register, and
                     mosaic thousands of high-resolution images with
                     minimal user intervention, resulting in a factor
                     of 10+ increase in analysis precision.
                   \item Created a genetic
                     algorithm in C++ to efficiently fit galaxy models with
                     unusually large numbers of
                     free parameters to high-resolution images.
                  \end{itemize} 

\vspace{-0.1in}
\section{\centerline{\large EDUCATION}}
PhD, Astronomy, University of Wisconsin-Madison \hfill\mbox{December 2013}\\
MS, Astronomy, University of Wisconsin-Madison \hfill\mbox{June 2009}\\
BS, Astronomy, Case Western Reserve University \hfill\mbox{May 2007}\\
\vspace{-0.3in}
 
\end{resume} 
\end{document}









