% LaTeX file for resume 
% This file uses the resume document class (res.cls)

\documentclass[11pt]{res}
\topmargin=-0.380in  % start text higher on the page
\setlength{\textheight}{9.75in} % increase text height to fit resume on 1 page
\newsectionwidth{0pt}  % So the text is not indented under section headings
\begin{document}  
\name{Andrew Schechtman-Rook}

\address{rook166@gmail.com --- (917)-836-4267}
                        
\begin{resume} 
% \vspace{-0.1in}
% \section{\centerline{\large SUMMARY}}
% \vspace*{0.15 in}\begin{itemize}\itemsep -2pt
% \addtolength{\itemindent}{-0.215in}
% %\centering
% \item 8+ years of experience with analysis of large, complex datasets.
% \item Strong programming, numerical analysis, and visualization skills.
% \item Extensive familiarity with linear and non-linear model fitting.
% \item Experience writing software for both parallel and distributed
%   environments.
% \item Able to work independently as well as part of a team.
% \item Practiced writer and speaker for technical and non-technical
%   audiences.
% \end{itemize}
\vspace{-0.1in}
\section{\centerline{\large SKILLS}}
                    {\bf Programming:} C++, Python (including
                    Matplotlib, Numpy, Scikit-learn, Scipy), shell scripting.\\
                      {\bf Databases and Web Design:} Django, HTML, MySQL.\\
                        {\bf Operating Systems:} Linux, Mac OS X.\\
                        {\bf Data Analysis:} Bootstrapping; genetic algorithms; image
                        processing and machine vision; interpolation;
                        linear and non-linear regression; Monte Carlo; numerical integration and
                        differentiation; parallel and distributed
                        computing; principal component analysis; rootfinding.\\
                     {\bf Oral Presentations:} Described advanced
                       analysis and modeling to both expert and
                       non-expert audiences through engaging lectures
                       and informal discussions.\\
                       {\bf Writing:} Published research in leading
                         scientific journals (full list
                         available upon request). 
\vspace{-0.1in}
\section{\centerline{\large PROFESSIONAL EXPERIENCE}}
{\bf Postdoctoral Research Associate}\hfill\mbox{2014-Present}\\
University of Wisconsin-Madison
                  \vspace* {0.01 in}\begin{itemize} \itemsep -2pt
                    %\addtolength{\itemindent}{-0.215in}
                    \item Improved agreement between numerical models
                      and astronomical data by up to 20\% using new
                      metrics implemented in highly optimized Python.
                    \item Built a fast Voronoi Tessellation algorithm
                      to adaptively bin images, preserving spatial
                      resolution while maximizing signal in images
                      with over one million pixels.
                    % \item Constructed visualization tools using
                    %   Matplotlib to
                    %   aid data processing efforts.
                    % \item Trained and mentored undergraduate and
                    %   graduate students in programming, data analysis and statistical methods.
                    \end{itemize}
\vspace{-0.15in}
{\bf Research Assistant}\hfill\mbox{2007-2013}\\
University of Wisconsin-Madison
                  \vspace* {0.01 in}\begin{itemize} \itemsep -2pt
                    %\addtolength{\itemindent}{-0.215in}
                    \item Developed a non-linear Levenberg-Marquardt
                      $\chi^{2}$ fitting algorithm using a
                      combination of Python and C++ to constrain
                      models of spiral galaxies to data. 
                   \item Employed on-campus distributed computing
                     resources to perform large-scale modeling in parallel, using over 20 years of computer time
                     in 1 month.
                   \item Assembled a hybrid C++/Python processing
                     pipeline to process hundreds of high-resolution images with
                     minimal user intervention, resulting in a 10x increase in analysis precision.
                   \item Created a genetic
                     algorithm in C++ to efficiently fit galaxy models with
                     unusually large numbers of
                     free parameters to high-resolution images.
                   \item Utilized frequency-domain analysis to
                     understand the spatial distribution of galactic structure.
                  \item Computed descriptive statistics about
                    200+ astronomical objects from raw survey data
                    automatically via custom-built analysis software
                    blending C++ and shell scripting programs.
                  \item  Classified a previously unknown
                    galaxy using data from seven different sources. 
                  \end{itemize} 
\vspace {-0.15 in}
% {\bf Teaching Assistant}\hfill\mbox{2008-2009}\\
% University of Wisconsin-Madison
% \vspace*{0.01 in}\begin{itemize}\itemsep -2pt
%   %\addtolength{\itemindent}{-0.215in}
% \item Taught six discussion sections of an introductory undergraduate
%   astronomy course.
% \item Prepared engaging lesson plans, including interactive demonstrations and
%   group problem-solving activities.
% \end{itemize}
% \vspace{-0.15 in}
{\bf Research Assistant}\hfill\mbox{2006-2007}\\
Case Western Reserve University
                  \vspace* {0.01 in}\begin{itemize} \itemsep -2pt
                    %\addtolength{\itemindent}{-0.215in}
                    \item Designed and executed statistical analyses
                      of simulated galaxy clusters to optimize
                      strategy for future data acquisition.
                    \item Used 
                      regression analysis to compute a conversion
                      between astronomical filter systems. 
                  \end{itemize} 
% \vspace {-0.15 in}
% {\bf Physics Lab Assistant}\hfill\mbox{2005-2006}\\
% Case Western Reserve University
% \vspace*{0.01 in}\begin{itemize} \itemsep -2pt
%   %\addtolength{\itemindent}{-0.215in}
% \item Maintained existing equipment and computers for introductory
%   physics labs.
% \item Developed and built components for new labs.
% \item Upgraded hardware and software for over 20 lab computers.
% \end{itemize}
\vspace{-0.1in}
\section{\centerline{\large OTHER RELEVANT EXPERIENCE}}
{\bf Independent NFL Analyst}\hfill\mbox{2013-Present}\\
phdfootball.blogspot.com
                  \vspace* {0.01 in}\begin{itemize} \itemsep -2pt
                    %\addtolength{\itemindent}{-0.215in}
                    %\item Performed novel statistical analyses on
                     % publicly available NFL data.
                    \item Mined a play-by-play database containing
                      over 500,000 records across dozens of tables
                      for complex relationships between individual players
                      as well as teams. 
                    \item Explained findings to a broad audience through both written posts and evocative figures.
                  \end{itemize} 
\vspace{-0.1in}
\section{\centerline{\large EDUCATION}}
Ph.D., Astronomy, University of Wisconsin-Madison \hfill\mbox{December 2013}\\
% \begin{itemize} \itemsep -2pt
% \item Wisconsin Space Grant Consortium Graduate Fellowship
% \item International Astronomical Union Travel Grant
% \item American Astronomical Society Chambliss Student Award
% \item University of Wisconsin-Madison Astronomy Department Whitford Award
% \end{itemize}
% \vspace{-0.15 in}
MS, Astronomy, University of Wisconsin-Madison \hfill\mbox{June 2009}\\
BS, Astronomy, Case Western Reserve University \hfill\mbox{May 2007} 
% \begin{itemize} \itemsep -2pt
% \item Graduated {\it cum laude}
% \item Minors in Physics and Classics
% \end{itemize}
% \vspace{-0.1in}
% \section{\centerline{\large SELECTED PUBLICATIONS}}
% {\bf Schechtman-Rook, A.} \& Bershady, M. A., ``Near-Infrared Structure
% of Fast and Slow Rotating Disk Galaxies'', 2014, {\it in prep}\\
% {\bf Schechtman-Rook, A.}, Ph.D. Dissertation: ``Lifting the Dusty
% Veil: Understanding the Stellar Structure of Spiral Disks'', 2013\\
% {\bf Schechtman-Rook, A.} \& Bershady, M. A., ``Near-infrared Detection
% of a Super-thin Disk in NGC 891'', 2013, {\it ApJ}, 773, 45\\
% {\bf Schechtman-Rook, A.} \& Hess, K. M., ``NGC 4656UV: A UV-selected
% Tidal Dwarf Galaxy Candidate'', 2012, {\it ApJ}, 750, 171\\
% {\bf Schechtman-Rook, A.}, Bershady, M. A., \& Wood, K., ``The
% Three-dimensional Distribution of Dust in NGC 891'', 2012, {\it ApJ}, 746,70

 
\end{resume} 
\end{document}









